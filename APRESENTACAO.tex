\newcommand{\subtitulo}[1]{\NoCaseChange{\textnormal{\break\Large\itshape#1}}}
\chapter*{Apresentação\smallskip\subtitulo{Um clássico da etnologia\\ sul-americanista}}
\markboth{Apresentação}{}
\addcontentsline{toc}{chapter}{Apresentação, \textit{por Peter Schröder}}

%\formular\itshape

\begin{flushright}
\textsc{peter schröder}\medskip
\end{flushright}

\textit{Die Aruaken}, livro raro que apenas pode ser encontrado em poucos
sebos, é uma síntese dos conhecimentos sobre os povos indígenas falantes
de línguas aruaque, ou \textit{aruak}/\,\textit{arawak}, no período antes da Primeira Guerra Mundial. É um trabalho clássico de suma importância para os
estudos comparativos dos povos aruaques, como pode ser observado pela
leitura de diversos artigos na coletânea organizada por Hill \&
Santos"-Granero, de 2002. Existe uma tradução não assinada para o
português, cujo manuscrito, datilografado em papel timbrado do
Ministério da Agricultura, estava depositado na biblioteca do \textsc{ppgas} do
Museu Nacional, vinculado à \textsc{ufrj}, até o incêndio do museu em 2 de setembro de 2018.

Segundo uma informação veiculada no site da Biblioteca Digital
Curt Nimuendajú (\textsc{bdcn}), a tradução teria sido encomendada por Roberto
Cardoso de Oliveira\footnote{1928--2006.} e foi realizada por Klaas Woortmann.\footnote{Blog Etnolinguistica, agosto de 2012. Die Aruaken --
  um clássico da etnologia sul"-americanista. Recuperado de:
  \textless{}https://hedra.com.br/r/A36\textgreater{};
  acesso em 13/04/2020.} O manuscrito original provavelmente virou
cinzas, mas há uma versão transcrita que pode ser baixada no website da
\textsc{bdcn}.\footnote{\textless{}https://hedra.com.br/r/pwg\textgreater{};
  acesso em 13/04/2020.} Além disso, em 2019, Nelson Sanjad, do Museu
Paraense Emilio Goeldi (\textsc{mpeg}), entrou em contato comigo comunicando que
ele tinha encontrado nos arquivos do museu o manuscrito de uma tradução
do livro para o português. Inicialmente tínhamos dúvidas sobre a
autoria do manuscrito, já que ela foi datilografada em papel timbrado do
antigo Ministério da Educação e Saúde (\textsc{mes}), mas uma comparação com a
versão transcrita disponível na \textsc{bdcn} revelou imediatamente que a autoria
é a mesma. Desse modo, podemos constatar que pelo menos uma perda
causada pelo incêndio do Museu Nacional não foi definitiva.

O livro \textit{Die Aruaken} é a segunda tese de doutorado de Max Schmidt, que já era doutor em Direito pela
Friedrich"-Alexander"-Universität Erlangen desde 1899 e tinha realizado
três expedições científicas na América do Sul, entre 1900--01, 1910 e 1914.
Depois de voltar da primeira expedição, em 1901, ele obteve emprego,
como assistente da diretoria, no Real Museu Etnológico em Berlim,
naquela época o centro dos estudos etnológicos americanistas na
Alemanha. Em 1916, Schmidt defendeu sua segunda tese na Faculdade de
Filosofia da Universidade de Leipzig. A publicação, tradicionalmente
obrigatória para receber definitivamente o título de doutor\footnote{Dr.\,phil. 
neste caso, segundo a credencial alemã.} no sistema universitário alemão, seguiu um ano depois.

Durante suas expedições, Schmidt já tinha observado a influência cultural
dos povos aruaques sobre grupos linguística e culturalmente
diferenciados, o que estimulou o interesse de tentar achar uma
explicação para a enorme expansão geográfica dos aruaques nas terras
baixas da América do Sul. Segundo Schmidt, o problema central do estudo
não seria descobrir a origem geográfica dos aruaques, mas explicar sua
dinâmica cultural. Na realidade, ele dedicou"-se às duas questões, porém
deu preferência à segunda. A forma como ele abordou esta questão de fato
indica caminhos para rumos posteriores da antropologia, enquanto a
primeira remete a interesses predominantes na etnologia alemã do século
\textsc{xix}. Apenas no quinto capítulo Schmidt aventa uma hipótese sobre a possível origem geográfica dos aruaques no sudoeste da Amazônia, operando com
especulações sobre contatos com as culturas do altiplano andino, como
\textit{tiwanaku}.

Chama a atenção que Schmidt problematiza diferenças entre classificações
linguísticas e características culturais e opera com distinções claras
entre fenômenos como língua e cultura e conceitos como aculturação,
difusão e mudança cultural em termos gerais. Seu argumento
epistemológico principal é que outros autores, anteriores a ele, não
teriam levantado as questões certas sobre a expansão dos aruaques e, por
isso, não teriam chegado a respostas satisfatórias. E a teoria de
Schmidt de fato é diferente daquelas dos antecessores, mostrando grande
originalidade para a época.

Nos comentários iniciais que precedem o texto principal,\footnote{O \textit{Methodologische Vorbemerkungen}, em alemão.} o autor explica seu
posicionamento teórico e metodológico: defende a interdisciplinaridade
e chama a própria abordagem de sociológica. Inicialmente, a posição com
relação à Doutrina dos Círculos Culturais\footnote{\textit{Kulturkreislehre}, em alemão.} de Fritz Graebner\footnote{1877--1934.} e do Padre Wilhelm Schmidt\footnote{1868--1954.} ainda pode ser descrita como reservada e cautelosa, porém ela se transforma em
rejeição contundente no final do trabalho. Do ponto de vista
metodológico, Max Schmidt, com sua defesa de comparações interculturais
sistemáticas e empiricamente fundamentadas, se posiciona mais próximo
dos ensinamentos de Franz Boas\footnote{1858--1942.} do que das teorias
difusionistas austro"-alemãs da época, refletindo as influências de
Adolf Bastian\footnote{1826--1905.} e Karl von den Steinen.\footnote{1855--1929.}

A estrutura do trabalho é muito clara: depois dos comentários
metodológicos e de um resumo sobre os estudos etnológicos realizados
sobre os povos aruaques até então, Schmidt escreve sobre os motivos,\footnote{No segundo capítulo.} os meios\footnote{No terceiro capítulo.} e o caráter\footnote{No quarto capítulo.} da expansão das
culturas aruaques. O quinto capítulo, o mais especulativo de todos, trata da
posição das culturas aruaques com relação a outras culturas --- indígenas
e não indígenas --- nas Américas, enquanto no sexto capítulo examina a
influência da expansão das culturas aruaques sobre as transformações de
várias manifestações culturais. Ao final, os resultados do estudo são
apresentados de forma concisa.

O caráter do estudo é etnológico, no sentido de uma comparação
sistemática de informações etnográficas. Para as análises
bibliográficas Schmidt lançou mão, além dos próprios trabalhos sobre os
\textit{paressí"-kabiší} ou \textit{paresí"-kabizi}, principalmente dos estudos de
autores como Paul Ehrenreich,\footnote{1855--1914.} Theodor Koch"-Grünberg,\footnote{1872--1924.} Erland Nordenskiöld,\footnote{1877--1932.} Karl von den Steinen\footnote{1855--1929.} e Everhard im Thurn,\footnote{1852--1932.} este com relação às Guianas. Em geral, as explicações e digressões etnográficas não são
exaustivas, mas têm a finalidade de dar sustento à teoria explicativa do
autor sobre a expansão social e cultural dos aruaques. A função delas é
basicamente ilustrativa.

O ponto de partida da análise de Schmidt é a identificação da
agricultura, combinada com uma maior complexidade social, como
denominador comum de todas as culturas aruaque, apesar de sua grande
diversidade em termos gerais. O autor destaca as culturas de mandioca e
milho como economicamente dominantes. Chama a atenção que as numerosas
variedades destas culturas agrícolas ainda não foram levadas em
consideração, enquanto elas são de suma relevância para os estudos
atuais de economias indígenas.

Desse modo, a abordagem \textit{sociológica} de Schmidt também podia ser
rotulada como \textit{socioeconômica}, embora não deva ser confundida com um
simples determinismo materialista: é muito diferente das teorias
predominantes na etnologia alemã da época. Schmidt apresenta uma cadeia
de consequências: 
\medskip

\textbf{Características do meio ambiente \& Agricultura} \rightarrow{} Sociedades sedentárias \rightarrow{} Necessidades crescentes por matéria"-prima \rightarrow{} Esgotamento de recursos naturais \rightarrow{} Aumento do trabalho a ser investido \rightarrow{} \textbf{Maior demanda por força de trabalho \& redes mais amplas
de troca e comércio}.

\medskip
Neste sentido, ele parece antecipar em parte a
ecologia cultural de Julian Steward.\footnote{1902--1972.}

Schmidt analisa as formas de organização social dos aruaques com
relação às atividades econômicas e às diferenciações sociais internas. A
expansão dos aruaques seria menos populacional, no sentido de grupos
inteiros se deslocarem para novos territórios, mas, sobretudo,
caracterizada por dominação social e cultural. A necessidade de manter
suas comunidades sedentárias desencadearia processos de procura por
ampliar a força de trabalho não mais encontrada na própria sociedade,
porém a incorporação de membros de outras sociedades se daria tanto por
subjugação militar quanto, majoritariamente, por influências culturais
exercidas de forma lenta e sutil, de modo que a expansão dos aruaques
possa ser chamada \textit{sociocultural}. Schmidt analisa as mais diversas
formas, relatadas nas etnografias consultadas, de relações entre
aruaques e não aruaques: conflitos interétnicos com o objetivo de roubar
ou escravizar mulheres e crianças, casamentos por rapto ou \textit{raubehe}, dominação militar, alianças políticas, casamentos
interétnicos pacificamente regularizados, adoções, visitas, festas,
rituais, trocas de objetos, etc. Até o ritual da couvade é interpretado
neste sentido: no caso de residência pós"-nupcial uxorilocal, sua função
seria socialmente agregativa por conseguir vincular genros à unidade
doméstica do sogro (aruaque).

Além desses mecanismos de dominação, Schmidt apresenta uma série de
outros que podiam ser rotulados de \textit{ideológicos} num sentido quase
marxista: mitos, determinados rituais e magias. Até as artes plásticas
são interpretadas como tendo essa função social.

A teoria de Schmidt sobre o caráter e as causas da expansão aruaque nas
terras baixas da América do Sul tem pouco a ver com as teorias
migratórias da época por identificar e explicitar diversos fatores
sociais e econômicos. Desse modo, ele conseguiu apresentar uma teoria
própria de mudança cultural, ao mesmo tempo funcionalista e dinâmica.
Ela pode ser mais bem caracterizada como uma teoria de sobreposição
social e cultural engatada com algum tipo de teoria de dependência
\textit{avant le nom}: os aruaques transformam em dependentes outros
grupos ou povos antes independentes por contribuir à satisfação das
necessidades econômicas destes e, ao mesmo tempo, de si mesmos. Com
as palavras do autor, ``Correspondem então, por um lado, o instinto de
ganho e, por outro lado, o instinto de subjugação''.\footnote{No original alemão: \textit{Es entsprechen sich also der Erwerbstrieb auf der einen Seite und der
Unterwerfungstrieb auf der anderen Seite}.} Um vocabulário
ultrapassado, sim, mas indicador de uma teoria original.

No sexto capítulo, Schmidt lança uma crítica contundente contra Padre Wilhelm
Schmidt e a \textit{doutrina dos círculos culturais}, em particular contra a
aplicação dessa teoria para explicar a diversidade das culturas
indígenas sul"-americanas.\footnote{Schmidt, 1913.} Tanto os círculos culturais\footnote{Em alemão, \textit{Kulturkreise}.} quanto as camadas culturais\footnote{Em alemão, \textit{Kulturschichten}.} não teriam nenhuma base empírica. O caráter
especulativo do difusionismo austro"-alemão é confrontado com a própria
teoria de mudanças culturais, inspirada, por sua vez, na teoria de
mudança cultural do sociólogo Alfred Vierkandt.\footnote{1867--1953.} Vierkandt distinguiu \textit{bens culturais},\footnote{Em alemão, \textit{Kulturgüter}.} \textit{essenciais} e \textit{não essenciais}, o que explica
sua utilidade para o modelo de mudança cultural esboçado por Schmidt
para os processos de {aruaquização}.\footnote{Em alemão, \textit{Aruakisierung}.}

\textit{Die Aruaken} é um clássico da etnologia sul"-americanista que ainda
vale a pena ler. Escrito num dos períodos mais terríveis da história
humana, é interessante ver que o livro termina com um apelo ao leitor
para que veja que muitas mudanças culturais podem ser alcançadas sem
imposições violentas. Será que foi uma lembrança tímida de um humanista
dos objetivos desastrosos do Império que levariam o país à derrota
militar e ao colapso econômico um ano depois?

\section{Sobre esta edição}

A ideia de publicar uma tradução de \textit{Die Aruaken} surgiu alguns
anos atrás numa conversa com Erik Petschelies. Ficamos pensando quais as
obras clássicas da tradição etnológica alemã com enfoque sul"-americanista
que ainda não foram traduzidas para o português. Imediatamente nos
lembramos de uma série de autores e publicações, mas optamos pelo livro
de Schmidt por causa de sua importância no contexto da etnologia
indígena das terras baixas da América do Sul. Inicialmente imaginamos
publicar apenas a tradução junto com uma apresentação e uma nota
explicativa do tradutor, mas com o tempo percebemos que uma
contextualização tanto biográfica quanto científica ajudará os leitores
a conhecer outros aspectos da obra e de seu autor.

Por isso, ficamos muito gratos ao saber que um colega da Universidade de
Göttingen, Michael Kraus, disponibilizou um artigo pronto para ser
publicado em inglês, mas que era inédito em língua portuguesa. Kraus é
especialista em história da antropologia e um dos melhores conhecedores
da história da etnologia alemã. O artigo de Kraus contextualiza não só o
livro traduzido, mas toda a obra de Schmidt como parte de uma tradição
na etnologia alemã e, ao mesmo tempo, explica suas particularidades e
seu caráter excepcional. Além disso, vale a pena ler o artigo, porque
ele desnuda, indiretamente, toda uma série de narrativas reducionistas e
simplórias que circulam em muitas grades curriculares nacionais de
graduação e pós"-graduação sobre autores clássicos como Schmidt, suas
visões dos ``outros'' e suas maneiras de conduzir pesquisas de campo. Em
outras palavras: o texto de Kraus merece ser lido não só por pessoas
interessadas na obra de Schmidt.

Também optamos por anexar dois breves textos biográficos, ambos de
1951, ou seja, publicados pouco tempo depois do falecimento de Schmidt:
o obituário redigido com muita sensibilidade por Herbert Baldus e um
texto quase desconhecido de Paulo de Carvalho Neto sobre os últimos ---
tristes --- dias de Schmidt. Max Schmidt faleceu em condições de miséria
total depois de ter dedicado, de modo incansável, uma vida inteira à
etnologia.

Aliás, a biografia acadêmica de Schmidt parece ser bem conhecida e
apenas a descoberta de documentos inéditos pode lançar novas luzes sobre
ela. Para quem gostaria de conhecê"-la em maiores detalhes, existem três
fontes principais: a \textit{autobiografia}, anotada e redigida por Carvalho
Neto,\footnote{Schmidt, 1955.} a sinopse biográfica de Susnik, de 1991, com
resenhas de todos os trabalhos de Schmidt, e, sobretudo, o artigo
recente de Bossert e Villar, de 2019. Pouco se sabe, no entanto, sobre a
biografia não acadêmica de Schmidt, porque sua personalidade sempre foi
descrita como muito reservada e tímida. Parece que Baldus não tinha
conhecimento nem do casamento com uma paraguaia chamada Mari, em 1914,
nem de alguns dilacerantes reveses trágicos da vida de Schmidt, porque
não os menciona no obituário.

Chega a ser um milagre que Schmidt conseguiu finalizar sua tese de
doutorado durante a Primeira Guerra Mundial, porque em maio de 1917
falecera sua mãe e em 24 de julho do mesmo ano, sua filha única, Grete.
Tudo indica que sua esposa já o tinha abandonado quando essas duas
tragédias se abateram sobre ele. Estas informações biográficas, nem
mencionadas no artigo de Bossert e Villar, apenas ficaram
conhecidas com a pesquisa documental criteriosa de Petschelies\footnote{2019, p.
521--529.} no espólio de Theodor Koch"-Grünberg, arquivado na
Philipps"-Universität Marburg.

Um dos aspectos mais enigmáticos de sua vida certamente são os motivos
de sua surpreendente renúncia de seus cargos acadêmicos na Alemanha, em
1929, e sua emigração, primeiro para o Brasil e depois para o Paraguai.
Bossert \& Villar\footnote{Páginas 23--24.} avançaram mais do que outros autores
por também levantar a hipótese de que Schmidt poderia ter tomado sua
decisão por pressentir o clima tóxico para a etnologia alemã com o
nazismo em ascensão na parte final da República de Weimar. Além disso,
parece que nem existiam mais vínculos familiares que poderiam ter
atrasado ou impedido sua emigração.

Seja como for, comparando as biografias de etnólogos alemães do período
do final do século \textsc{xix} até a década de 1940, chamam a atenção dois
aspectos semelhantes nas vidas de Schmidt e do antropólogo brasileiro de
origem alemã Curt Nimuendajú:\footnote{1883--1945.} a opção radical pela
emigração, sem as menores intenções de retorno, e a renúncia total a
qualquer conforto material. Como se sabe, as obras dos dois tiveram
impactos muito diferentes e Nimuendajú continua ser considerado uma
figura pioneira na constituição da etnologia indígena no Brasil e da
antropologia brasileira em geral, enquanto Schmidt é um autor citado
principalmente entre especialistas, embora a publicação de suas
fotografias por Bossert e Villar, com apoio do ator Viggo
Mortensen, certamente tenha ajudado a chamar a atenção para sua obra.

Pensamos que está na hora de fazer uma nova leitura, devidamente
contextualizada, de um autor que não mereceria ficar esquecido e cuja
obra \textit{Os Aruaques} continua ser um clássico, porque estava à
frente de sua época.

% \begin{bibliohedra}
% \tit{BOSSERT}, Federico \& \textsc{villar}, Diego. \textit{Hijos de la selva. La
% fotografía etnográfica de Max Schmidt -- Sons of the Forest. The
% Ethnographic Photography of Max Schmidt.} Santa Monica, \textsc{ca}: Perceval
% Press, 2013.

% \titidem. Una vida antropológica: biografía de Max Schmidt.
% \textit{Bérose -- Encyclopédie internationale des histoires de
% l'anthropologie.} Paris: \textsc{iiac"-lahic}, \textsc{cnrs}/Ministère de la Culture, 2019.
% (disponível em:
% \textless{}https://hedra.com.br/r/N3C\textgreater{};
% acesso em 15/04/2020)

% \tit{HILL}, Jonathan \& \textsc{santos-granero}, Fernando. \textit{Comparative
% Arawakan Histories: Rethinking Language Family and Culture Area in
% Amazonia\textit{.}} Urbana, Chicago: University of Illinois Press, 2002.

% \tit{PETSCHELIES}, Erik. \textit{As redes da etnografia alemã no Brasil
% (1884--1929).} (Tese de doutorado) Programa de Pós"-Graduação em
% Antropologia Social (\textsc{ppgas}), Universidade Estadual de Campinas,
% Campinas, 2019.

% \tit{SCHMIDT}, Max. Autobiografia de Max Schmidt. \textit{Revista de
% Antropologia,} São Paulo, v.\,3, n. 2, p. 115--124, 1955. (disponível em:
% \textless{}https://hedra.com.br/r/P4n\textgreater{};
% acesso em 15/04/2020)

% \tit{SCHMIDT}, Wilhelm, S.V.D. Kulturkreise und Kulturschichten in Südamerika.
% \textit{Zeitschrift für Ethnologie}, Berlin, n. 45, p. 1014--1030,
% 1913.

% \tit{SUSNIK}, Branislava. \textit{Prof. Dr.\,Max Schmidt: su contribución
% etnológica e su personalidad.} Asunción: Museo Etnográfico Andrés
% Barbero, 1991.

% \tit{VIERKANDT}, Alfred. \textit{Die Stetigkeit im Kulturwandel: Eine
% soziologische Studie}. Leipzig: Duncker \& Humblot, 1908.
% \end{bibliohedra}