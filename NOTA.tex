\chapter*{Nota sobre a tradução}
\markboth{Nota sobre a tradução}{}
\addcontentsline{toc}{chapter}{Nota sobre a tradução, \emph{por Erik Petschelies}}

\begin{flushright}
\textsc{erik petschelies}
\end{flushright}

\noindent{}Para essa edição em português da obra \emph{Die Aruaken}, buscou"-se
realizar uma tradução\footnote{Tradução realizada com apoio da Fundação de Amparo à Pesquisa do Estado de São Paulo (\textsc{fapesp}), processo 2019/18641-9. ``As opiniões, hipóteses e conclusões ou recomendações expressas neste material são de responsabilidade do(s) autor(es) e não necessariamente refletem a visão da \textsc{fapesp}.''} que contemplasse o registro narrativo de Max
Schmidt, marcado por um vocabulário às vezes arcaico e um estilo
notoriamente prolixo. Nesse sentido, dois apontamentos são necessários.

O primeiro diz respeito à grafia dos nomes dos grupos e das línguas
indígenas. No corpo do texto os nomes foram grafados de acordo com a
convenção da \textsc{aba}, de 1976, que constituiu um padrão para a sua
nomenclatura. No entanto, considerando que a forma como eles foram
anotados por Max Schmidt demonstra uma certa relação dos grupos
indígenas com o Estado e com alguns dos seus vetores, uma mera
substituição das suas grafias submeteria uma história política inerente
à nomenclatura ao esquecimento. Assim, há ao final dessa nota uma lista
em que os nomes indígenas aparecem tal como grafados por Max Schmidt em
1917 e como o são na presente tradução.

O segundo apontamento está estritamente relacionado ao primeiro.
Optou"-se por manter a versão original do famoso mapa da expansão das
culturas aruaque, pois uma atualização acarretaria, provavelmente, uma
perda de seu valor histórico e etnológico. Dessa maneira, a lista dos
nomes deve servir como guia para a compreensão do mapa, que assim
mantém, por um lado, seu valor documental para a história da
antropologia e dos índios, e, por outro, revela o estado das relações
sociais em que os povos indígenas estavam inseridos no momento de sua
confecção.

Os nomes dos grupos e das línguas são apresentados no original, em
alemão, seguidos pelas versões em português. Algumas equivalências se
devem por sua própria manutenção, outros, porque o registro de Max
Schmidt deve ter sido um dos últimos antes da extinção de determinado
povo e as demais, por fim, pois não se encontrou mais nenhuma menção aos
povos, exceto aquelas do próprio texto.

% Agradeço a Peter Schröder pela sua participação nesta empreitada
% iniciada há oito anos e por revisar tão cuidadosamente minha tradução, a
% Luisa Valentini pelo apoio e à editora Hedra pela publicação. Por fim,
% sou grato a Héllen Bezerra pelo seu suporte incondicional e seu
% companheirismo inspirador.

\begin{center}
\begin{tabular}{ | m{11em} | m{4.2cm}| } 
\hline
\textbf{Alemão} & \textbf{Português} \\ [0.5ex] 
\hline\hline
Alte(n)/\,Frühe(n) Aruaken & Antigo(s) Aruaque(s) \\
\hline
Anti 						& Anti \\
\hline
Arauití 					& Arauiti \\
\hline
Araycu/\,Uraycu              & Araycu/\,Uraycu  \\
\hline
Arrua           	  	       	  & Aruá \\
\hline
Aruaken/\,Arowaken 	       	  & Aruaque(s) \\
\hline
Aueotó/\,Auetö    	       	  & Aweti \\
\hline
Bakairi           	       	  & Bakairi \\
\hline
Baniva            	       	  & Baniwa \\
\hline
Baré              	       	  & Baré \\
\hline
Bauré             	       	  & Bauré \\
\hline
Betoya            	       	  & Betoya \\
\hline
\end{tabular}
\end{center}

\pagebreak



\begin{center}
\begin{tabular}{ | m{11em} | m{4.2cm}| } 
\hline
\textbf{Alemão} & \textbf{Português} \\ [0.5ex] 
\hline\hline
Cauixana          	       	  & Cauixana \\
\hline
Chané             	       	  & Chané \\
\hline
Chiriguano        	       	  & Chiriguano \\
\hline
Desana            	       	  & Desana \\
\hline
Ges               	       	  & Jê \\
\hline
Goajiro           	       	  & Guajiro \\
\hline
Guaiguakuré       	       	  & Guaiguacuré \\
\hline
Guaná             	       	  & Guaná \\
\hline
Guaraní           	       	  & Guarani \\
\hline
Guató             	       	  & Guató \\
\hline
Hölöua            	       	  & Hölöua \\
\hline
Huhuteni          	       	& Huhuteni \\
\hline
Ipuriná           	       	& Apurinã \\
\hline
Jukuna            	       	& Yukuna \\
\hline
Juri              	       	& Juri \\
\hline
Kaingua           	       	& Kaingua \\
\hline
Kajabí            	       	& Kaiabi \\
\hline
Kamayurá          	       	& Kamaiurá \\
\hline
Kampa             	       	& Campa/\,Kampa/\,Askaninka \\
\hline
Karaiben          	       	& Karib \\
\hline
Kaua              	       	& Kawá-Tapuya \\
\hline
Kaxiniti          	       	& Kaxiniti \\
\hline
Kobeua            	       	& Kubeo \\
\hline
Kustenaú          	       	& Kustenau \\
\hline
Maipuré           	       	& Maipuré \\
\hline
Makú              	       	& Maku \\
\hline
Makuši            	       	& Makuxi \\
\hline
Manaos            	       	& Manaó \\
\hline
Maraua            	       	& Maraua \\
\hline
Mbajá             	       	& Mbayá \\
\hline
Mehinakú          	       	& Mehinakú \\
\hline
\end{tabular}
\end{center}


\pagebreak

\begin{center}
\begin{tabular}{ | m{11em} | m{4.2cm}| } 
\hline
\textbf{Alemão} & \textbf{Português} \\ [0.5ex] 
\hline\hline
Mojo              	       	& Mojo \\
\hline
Nahukuá           	       	& Nahukuá \\
\hline
Nahukuhá          	       	& Nahukuhá \\
\hline
Paressí           	       	& Paresí \\
\hline
Paressí-Kabiší    	       	& Paresí-Kabizi \\
\hline
Passé             	       	& Passé \\
\hline
Pauišana          	       	& Pauisana \\
\hline
Paumari           	       	& Paumari \\
\hline
Piratapuya        	       	& Pira-tapuya \\
\hline
Piro              	       	& Piro \\
\hline
Purupurú          	       	& Purupuru \\
\hline
Purús             	       	& Purús \\
\hline
Siriono           	       	& Siriono \\
\hline
Siusi             	       	& Siusi \\
\hline
Suyá              	       	& Suyá \\
\hline
Tariana           	       	& Tariana \\
\hline
Tereno            	       	& Terena \\
\hline
Tiahuanoco        	       	& Tiahuanoco \\
\hline
Toba              	       	& Toba \\
\hline
Trumai            	       	& Trumai \\
\hline
Tšamakoko         	       	& Chamacoco \\
\hline
Tukano            	       	& Tukano \\
\hline
Tupi              	       	& Tupi \\
\hline
Uaupé             	       	& Waupé \\
\hline
Ueimaré           	       	& Ueimaré \\
\hline
Wapisiana         	       	& Wapixana \\
\hline
Waurá             	       	& Waurá \\
\hline
Yamamadí          	       	& Jamamadi \\
\hline
Yauaperí          	       	& Jauaperi \\
\hline
Yaulapíti         	       	& Yawalapiti \\
\hline
Yurupary-tapuya   	       	& Yurupary-tapuya\\
\hline
\end{tabular}
\end{center}

%\endgroup
