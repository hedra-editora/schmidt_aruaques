\chapter*{Nota sobre a tradução}
\markboth{Nota sobre a tradução}{}
\addcontentsline{toc}{chapter}{Nota sobre a tradução, \emph{por Erik Petschelies}}

\begin{flushright}
\textsc{erik petschelies}
\end{flushright}

Para essa edição em português da obra \emph{Die Aruaken}, buscou"-se
realizar uma tradução\footnote{Tradução realizada com apoio da Fundação de Amparo à Pesquisa do Estado de São Paulo (\textsc{fapesp}), processo 2019/18641-9. ``As opiniões, hipóteses e conclusões ou recomendações expressas neste material são de responsabilidade do(s) autor(es) e não necessariamente refletem a visão da \textsc{fapesp}.''} que contemplasse o registro narrativo de Max
Schmidt, marcado por um vocabulário às vezes arcaico e um estilo
notoriamente prolixo. Nesse sentido, dois apontamentos são necessários.

O primeiro diz respeito à grafia dos nomes dos grupos e das línguas
indígenas. No corpo do texto os nomes foram grafados de acordo com a
convenção da \textsc{aba}, de 1976, que constituiu um padrão para a sua
nomenclatura. No entanto, considerando que a forma como eles foram
anotados por Max Schmidt demonstra uma certa relação dos grupos
indígenas com o Estado e com alguns dos seus vetores, uma mera
substituição das suas grafias submeteria uma história política inerente
à nomenclatura ao esquecimento. Assim, há ao final dessa nota uma lista
em que os nomes indígenas aparecem tal como grafados por Max Schmidt em
1917 e como o são na presente tradução.

O segundo apontamento está estritamente relacionado ao primeiro.
Optou"-se por manter a versão original do famoso mapa da expansão das
culturas aruaque, pois uma atualização acarretaria, provavelmente, uma
perda de seu valor histórico e etnológico. Dessa maneira, a lista dos
nomes deve servir como guia para a compreensão do mapa, que assim
mantém, por um lado, seu valor documental para a história da
antropologia e dos índios, e, por outro, revela o estado das relações
sociais em que os povos indígenas estavam inseridos no momento de sua
confecção.

Os nomes dos grupos e das línguas são apresentados no original, em
alemão, seguidos pelas versões em português. Algumas equivalências se
devem por sua própria manutenção, outros, porque o registro de Max
Schmidt deve ter sido um dos últimos antes da extinção de determinado
povo e as demais, por fim, pois não se encontrou mais nenhuma menção aos
povos, exceto aquelas do próprio texto.

% Agradeço a Peter Schröder pela sua participação nesta empreitada
% iniciada há oito anos e por revisar tão cuidadosamente minha tradução, a
% Luisa Valentini pelo apoio e à editora Hedra pela publicação. Por fim,
% sou grato a Héllen Bezerra pelo seu suporte incondicional e seu
% companheirismo inspirador.

\begin{table}[]
\begin{tabular}{|l|l|}
\textbf{Original alemão}   & \textbf{Português atual}                          \\
Alte(n)/\,Frühe(n) Aruaken & Antigos Aruaque(s)                             \\
Anti                       & Anti                                           \\
Arauití                    & Arauiti \textbackslash{}\textbackslash{}       \\
Araycu/Uraycu              & Araycu/\,Uraycu \textbackslash{}\textbackslash{} \\
Arrua                      & Aruá \textbackslash{}\textbackslash{}          \\
Aruaken, Arowaken          & Aruaque(s) \textbackslash{}\textbackslash{}    \\
Aueotó / Auetö             & Aweti \textbackslash{}\textbackslash{}         \\
Bakairi                    & Bakairi \textbackslash{}\textbackslash{}       \\
Baniva                     & Baniwa \textbackslash{}\textbackslash{}        \\
Baré                       & Baré \textbackslash{}\textbackslash{}          \\
Bauré                      & Bauré \textbackslash{}\textbackslash{}         \\
Betoya                     & Betoya \textbackslash{}\textbackslash{}        \\
Cauixana                   & Cauixana \textbackslash{}\textbackslash{}      \\
Chané                      & Chané \textbackslash{}\textbackslash{}         \\
Chiriguano                 & Chiriguano \textbackslash{}\textbackslash{}    \\
Desana                     & Desana \textbackslash{}\textbackslash{}        \\
Ges                        & Jê \textbackslash{}\textbackslash{}            \\
Goajiro                    & Guajiro \textbackslash{}\textbackslash{}       \\
Guaiguakuré                & Guaiguacuré \textbackslash{}\textbackslash{}   \\
Guaná                      & Guaná \textbackslash{}\textbackslash{}         \\
Guaraní                    & Guarani \textbackslash{}\textbackslash{}       \\
Guató                      & Guató \textbackslash{}\textbackslash{}         \\
Hölöua                     & Hölöua \textbackslash{}\textbackslash{}        \\
Huhuteni                   & Huhuteni \textbackslash{}\textbackslash{}      \\
Ipuriná                    & Apurinã \textbackslash{}\textbackslash{}      
\end{tabular}
\end{table}

\begin{table}[]
\begin{tabular}{|l|l|}
\textbf{Original alemão} & \textbf{Português atual}\\
Jukuna                   & Yukuna \textbackslash{}\textbackslash{}\\
Juri                     & Juri \textbackslash{}\textbackslash{}\\
Kaingua                  & Kaingua \textbackslash{}\textbackslash{}\\
Kajabí                   & Kaiabi \textbackslash{}\textbackslash{}\\
Kamayurá                 & Kamaiurá \textbackslash{}\textbackslash{}\\
Kampa                    & Campa/\,Kampa/\,Askaninka \textbackslash{}\textbackslash\\
Kobeua                   & Kubeo \textbackslash{}\textbackslash{}\\
Karaiben                 & Karib \textbackslash{}\textbackslash{}\\
Kaua                     & Kawá-Tapuya \textbackslash{}\textbackslash{}\\
Kaxiniti                 & Kaxiniti \textbackslash{}\textbackslash{}\\
Kustenaú                 & Kustenau \textbackslash{}\textbackslash{}\\
Maipuré                  & Maipuré \textbackslash{}\textbackslash{}\\
Makú                     & Maku \textbackslash{}\textbackslash{}\\
Makuši                   & Makuxi \textbackslash{}\textbackslash{}\\
Manaos                   & Manaó \textbackslash{}\textbackslash{}\\
Maraua                   & Maraua \textbackslash{}\textbackslash{}\\
Mbajá                    & Mbayá \textbackslash{}\textbackslash{}\\
Mehinakú                 & Mehinakú \textbackslash{}\textbackslash{}\\
Mehináku                 & Mehináku \textbackslash{}\textbackslash{}\\
Mojo                     & Mojo \textbackslash{}\textbackslash{}\\
Nahukuá                  & Nahukuá \textbackslash{}\textbackslash{}\\
Nahukuhá                 & Nahukuhá \textbackslash{}\textbackslash{}\\
Paressí                  & Paresí \textbackslash{}\textbackslash{}\\
Paressí-Kabiší           & Paresí-Kabizi \textbackslash{}\textbackslash{}\\
Passé                    & Passé \textbackslash{}\textbackslash{}\\
Pauišana                 & Pauisana \textbackslash{}\textbackslash{}      
\end{tabular}
\end{table}

\begin{table}[]
\begin{tabular}{|l|l|}
\textbf{Original alemão} & \textbf{Português atual}                            \\
Paumari                  & Paumari \textbackslash{}\textbackslash{}         \\
Piratapuya               & Pira-tapuya \textbackslash{}\textbackslash{}     \\
Piro                     & Piro \textbackslash{}\textbackslash{}            \\
Purupurú                 & Purupuru \textbackslash{}\textbackslash{}        \\
Purús                    & Purús \textbackslash{}\textbackslash{}           \\
Siriono                  & Siriono \textbackslash{}\textbackslash{}         \\
Siusi                    & Siusi \textbackslash{}\textbackslash{}           \\
Suyá                     & Suyá \textbackslash{}\textbackslash{}            \\
Tariana                  & Tariana \textbackslash{}\textbackslash{}         \\
Tereno                   & Terena \textbackslash{}\textbackslash{}          \\
Tiahuanoco               & Tiahuanoco \textbackslash{}\textbackslash{}      \\
Toba                     & Toba \textbackslash{}\textbackslash{}            \\
Trumai                   & Trumai \textbackslash{}\textbackslash{}          \\
Tšamakoko                & Chamacoco \textbackslash{}\textbackslash{}       \\
Tukano                   & Tukano \textbackslash{}\textbackslash{}          \\
Tupi                     & Tupi \textbackslash{}\textbackslash{}            \\
Uaupé                    & Waupé \textbackslash{}\textbackslash{}           \\
Ueimaré                  & Ueimaré \textbackslash{}\textbackslash{}         \\
Wapisiana                & Wapixana \textbackslash{}\textbackslash{}        \\
Waurá                    & Waurá \textbackslash{}\textbackslash{}           \\
Yamamadí                 & Jamamadi \textbackslash{}\textbackslash{}        \\
Yauaperí                 & Jauaperi \textbackslash{}\textbackslash{}        \\
Yaulapíti                & Yawalapiti \textbackslash{}\textbackslash{}      \\
Yurupary-tapuya          & Yurupary-tapuya \textbackslash{}\textbackslash{}
\end{tabular}
\end{table}