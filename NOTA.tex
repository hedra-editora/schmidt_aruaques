\chapter[{[}\textsc{n.\,t.}{]}, \emph{por Erik Petschelies}]{{[}\textsc{n.\,t.}{]}\footnote{Tradução realizada com apoio da Fundação de Amparo à Pesquisa do Estado de São Paulo (\textsc{fapesp}), processo 2019/18641-9.
``As opiniões, hipóteses e conclusões ou recomendações expressas neste material são de responsabilidade do(s) autor(es) e não necessariamente refletem a visão da \textsc{fapesp}.''}}
\hedramarkboth{{[}\textsc{n.\,t.}{]}}{}

\begin{flushright}
\textsc{erik petschelies}\medskip
\end{flushright}

\noindent{}Para essa edição em português da obra \emph{Die Aruaken} buscou"-se
realizar uma tradução que contemplasse o registro narrativo de Max
Schmidt, marcado por um vocabulário às vezes arcaico e um estilo
notoriamente prolixo. Nesse sentido, dois apontamentos são necessários.
O primeiro diz respeito à grafia dos nomes dos grupos e das línguas
indígenas. No corpo do texto os nomes foram grafados de acordo com a
convenção da \textsc{aba} de 1976 que constituiu um padrão para a sua
nomenclatura. No entanto, considerando que a forma como eles foram
anotados por Max Schmidt demonstra uma certa relação dos grupos
indígenas com o Estado e com alguns dos seus vetores, uma mera
substituição das suas grafias submeteria uma história política inerente
à nomenclatura ao esquecimento. Assim, há ao final dessa nota uma lista
em que os nomes indígenas aparecem tal como grafados por Max Schmidt em
1917 e como o são na presente tradução.

O segundo apontamento está estritamente relacionado ao primeiro.
Optou"-se por manter a versão original do famoso mapa da expansão das
culturas aruaque, pois uma atualização acarretaria, provavelmente, uma
perda de seu valor histórico e etnológico. Dessa maneira, a lista dos
nomes deve servir como guia para a compreensão do mapa, que assim
mantém, por um lado, seu valor documental para a história da
antropologia e dos índios, e, por outro, revela o estado das relações
sociais em que os povos indígenas estavam inseridos no momento de sua
confecção.

Os nomes dos grupos e das línguas são apresentados no original, em
alemão, seguidos pelas versões em português. Algumas equivalências se
devem por sua própria manutenção, outros, porque o registro de Max
Schmidt deve ter sido um dos últimos antes da extinção de determinado
povo e as demais, por fim, pois não se encontrou mais nenhuma menção aos
povos, exceto aquelas do próprio texto.

Agradeço a Peter Schröder pela sua participação nesta empreitada
iniciada há oito anos e por revisar tão cuidadosamente minha tradução, a
Luisa Valentini pelo apoio e à editora Hedra pela publicação. Por fim,
sou grato a Héllen Bezerra pelo seu suporte incondicional e seu
companheirismo inspirador.

\begin{table}[ht!]
\begin{center}
\begin{tabular}{ll}
\textbf{No original em alemão} & \textbf{Em português atual} \\
Alte(n) / Frühe(n) Aruaken     & antigos Aruaque(s)          \\
Anti                           & Anti       				\\                
Arauití & Arauiti \\ 
Araycu/Uraycu & Araycu/Uraycu \\ 
Arrua & Aruá \\ 
Aruaken, Arowaken & Aruaque(s) \\ 
Aueotó / Auetö & Aweti \\ 
Bakairi & Bakairi \\ 
Baniva & Baniwa \\ 
Baré & Baré \\ 
Bauré & Bauré \\ 
Betoya & Betoya \\ 
Cauixana & Cauixana \\ 
Chané & Chané \\ 
Chiriguano & Chiriguano \\ 
Desana & Desana \\ 
Ges & Jê \\ 
Goajiro & Guajiro \\ 
Guaiguakuré & Guaiguacuré \\ 
Guaná & Guaná \\ 
\end{tabular}
\end{center}
\end{table}

\begin{table}[ht!]
\begin{center}
\begin{tabular}{ll}
\textbf{No original em alemão} & \textbf{Em português atual} \\
Guaraní & Guarani \\ 
Guató & Guató \\ 
Hölöua & Hölöua \\ 
Huhuteni & Huhuteni \\ 
Ipuriná & Apurinã \\ 
Jukuna & Yukuna \\ 
Juri & Juri \\ 
Kaingua & Kaingua \\ 
Kajabí & Kaiabi \\ 
Kamayurá & Kamaiurá \\ 
Kampa & Campa/Kampa/Askaninka \\ 
Karaiben & Karib \\ 
Kaua & Kawá-Tapuya \\ 
Kaxiniti & Kaxiniti \\ 
Kobeua & Kubeo \\ 
Kustenaú & Kustenau \\ 
Maipuré & Maipuré \\ 
Makú & Maku \\ 
Makuši & Makuxi \\ 
Manaos & Manaó \\ 
Maraua & Maraua \\ 
Mbajá & Mbayá \\ 
Mehinakú & Mehinakú \\ 
Mehináku & Mehináku \\ 
Mojo & Mojo \\ 
Nahukuá & Nahukuá \\ 
Nahukuhá & Nahukuhá \\ 
Paressí & Paresí \\ 
Paressí-Kabiší & Paresí-Kabizi \\ 
Passé & Passé \\ 
Pauišana & Pauisana \\ 
\end{tabular}
\end{center}
\end{table}

\begin{table}[ht!]
\begin{center}
\begin{tabular}{ll}
\textbf{No original em alemão} & \textbf{Em português atual} \\
Paumari & Paumari \\ 
Piratapuya & Pira-tapuya \\ 
Piro & Piro \\ 
Purupurú & Purupuru \\ 
Purús & Purús \\ 
Siriono & Siriono \\ 
Siusi & Siusi \\ 
Suyá & Suyá \\ 
Tariana & Tariana \\ 
Tereno & Terena \\ 
Tiahuanoco & Tiahuanoco \\ 
Toba & Toba \\ 
Trumai & Trumai \\ 
Tšamakoko & Chamacoco \\ 
Tukano & Tukano \\ 
Tupi & Tupi \\ 
Uaupé & Waupé \\ 
Ueimaré & Ueimaré \\ 
Wapisiana & Wapixana \\ 
Waurá & Waurá \\ 
Yamamadí & Jamamadi \\ 
Yauaperí & Jauaperi \\ 
Yaulapíti & Yawalapiti \\ 
Yurupary-tapuya & Yurupary-tapuya \\ 
\end{tabular}
\end{center}
\end{table}