\textbf{Max Schmidt}, filho de jurista, nascido em 1874 em Hamburgo, na Alemanha, estudou ciências jurídicas e doutorou-se em direito romano na Universidade de Erlangen em 1899. Entre este ano e 1929, trabalhou no Museu de Antropologia de Berlim, primeiro como voluntário, depois como assistente de direção e, por fim, como chefe da seção americanista. Em 1916, defendeu seu segundo doutorado, desta vez em antropologia, com a tese ``Os Aruaques''. No ano seguinte, tornou-se professor da Universidade de Berlim. Em 1929, mudou-se definitivamente para a América do Sul, inicialmente para o Brasil e em seguida para a capital paraguaia Assunção. Ali lecionou na Escuela Superior de Filosofía, e dirigiu entre 1934 e 1946 o Museo Etnográfico Andrés Barbero. Schmidt foi um dos pioneiros da etnologia sul-"americanista, ao realizar três expedições ao Brasil Central e ao Pantanal (em 1900, 1910 e 1926--1828), além de várias pequenas incursões de campo durante sua estadia no Paraguai. Ele publicou artigos e livros sobretudo nos âmbitos da etnologia indígena, antropologia jurídica e econômica. Faleceu em Assunção, no ano de 1950.

\textbf{Os aruaques} é um livro clássico, escrito antes da Primeira Guerra Mundial, sobre os povos indígenas falantes de línguas aruaque. Durante suas expedições Max Schmidt já tinha observado a influência cultural dos povos aruaques sobre grupos diferenciados, além de sua enorme expansão geográfica nas terras baixas da América do Sul. O problema central não era descobrir a origem geográfica dos aruaques, mas explicar sua dinâmica cultural. Schmidt opera com distinções claras entre fenômenos como língua e cultura e conceitos como aculturação, difusão e mudança cultural em termos gerais. Seu argumento principal é que outros autores, anteriores a ele, não teriam levantado as questões certas sobre a expansão dos aruaques, por isso a falta de respostas satisfatórias. Sua teoria de fato é diferente dos antecessores, mostrando grande originalidade para a época.

\pagebreak

\textbf{Erik Petschelies} nasceu em Reinbek, Alemanha, em 1986. É bacharel em Ciências Sociais, mestre e doutor em Antropologia, sempre pela Universidade Estadual de Campinas (Unicamp). Entre 2016 e 2017, foi pesquisador-visitante da Philipps-"Universität Marburg na Alemanha. Atualmente faz seu pós-doutorado em Antropologia na Universidade de São Paulo (\textsc{usp}). Empreende pesquisas em História e Teoria da Antropologia, Historiografia da Ciência e História dos Povos Indígenas. É casado e pai de um filho.

\textbf{Peter Schröder} nasceu em Hannover, Alemanha, em 1960. Formou-se em etnologia pelas universidades de Marburg, Köln e Bonn, e doutorou-se em 1993 pela Rheinische Friedrich-Wilhelms-Universität. Professor associado no programa de pós-graduação em Antropologia (\textsc{ppga}) da Universidade Federal de Pernambuco (\textsc{ufpe}), em Recife, é também pesquisador do \textsc{cnp}q. Pesquisa os povos indígenas Guajajara e Fulni-ô, movimentos políticos indígenas, antropologia do desenvolvimento e história da antropologia. Desde 2009, pesquisa também as relações entre antropologias alemã e brasileira, focado na vida e obra do etnólogo Curt Nimuendajú.
% Excluído: do departamento de Antropologia e Museologia (\textsc{dam}).

\textbf{Coleção Mundo Indígena} \lipsum[2]


