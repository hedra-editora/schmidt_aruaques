\textbf{Max Schmidt} (Hamburgo, Alemanha, 1874--Assunção, Paraguai, 1950), filho de jurista, estudou ciências jurídicas e doutorou-se em direito romano na Universidade de Erlangen, em 1899. Entre este ano e 1929, trabalhou no Museu de Antropologia de Berlim, primeiro como voluntário, depois como assistente de direção e, por fim, como chefe da seção americanista. Em 1916, defendeu seu segundo doutorado, desta vez em antropologia, com a tese ``Os aruaques''. No ano seguinte, tornou-se professor da Universidade de Berlim. Em 1929, mudou-se definitivamente para a América do Sul, inicialmente para o Brasil e em seguida para a capital paraguaia Assunção. Ali lecionou na Escuela Superior de Filosofía, e dirigiu entre 1934 e 1946 o Museo Etnográfico Andrés Barbero. Schmidt foi um dos pioneiros da etnologia sul-americanista, ao realizar três expedições ao Brasil central e ao Pantanal (em 1900, 1910 e 1926--1828), além de várias pequenas incursões de campo durante sua estadia no Paraguai. Publicou artigos e livros sobretudo nos âmbitos da etnologia indígena, antropologia jurídica e econômica. Faleceu em Assunção,~no~ano~de~1950.

\textbf{Os aruaques}, publicado em 1917, é um clássico da etnologia sul-americanista. Escrito durante o período da Primeira Guerra Mundial, apresenta uma análise dos povos indígenas falantes de línguas aruaque. Durante suas expedições, Max Schmidt já tinha observado a influência cultural dos povos aruaques sobre outros grupos, além de sua enorme expansão espacial pelas terras baixas da América do Sul. No entanto, propõe explicar não sua origem geográfica, mas a dinâmica cultural. Schmidt usa de distinções claras entre fenômenos linguísticos e culturais ao longo do livro, além de conceitos específicos como \textit{aculturação}, \textit{difusão} e \textit{mudança cultural}.

\pagebreak
\thispagestyle{empty}

\textbf{Erik Petschelies} nasceu em Reinbek, Alemanha, em 1986. É bacharel em Ciências Sociais, mestre e doutor em Antropologia, sempre pela Universidade Estadual de Campinas (Unicamp). Entre 2016 e 2017, foi pesquisador-visitante da Philipps-Universität Marburg na Alemanha. Atualmente faz seu pós-doutorado em Antropologia na Universidade de São Paulo (\textsc{usp}). Empreende pesquisas em História e Teoria da Antropologia, Historiografia da Ciência e História dos Povos Indígenas. É casado e pai de um filho.

\textbf{Peter Schröder} nasceu em Hannover, Alemanha, em 1960. Formou-se em etnologia pelas universidades de Marburg, Köln e Bonn, e doutorou-se em 1993 pela Rheinische Friedrich-Wilhelms-Universität. Professor associado no programa de pós-graduação em Antropologia (\textsc{ppga}) da Universidade Federal de Pernambuco (\textsc{ufpe}), em Recife, é também pesquisador do \textsc{cnp}q. Pesquisa os povos indígenas Guajajara e Fulni-ô, movimentos políticos indígenas, antropologia do desenvolvimento e história da antropologia. Desde 2009, pesquisa também as relações entre antropologias alemã e brasileira, focado na vida e obra do etnólogo Curt Nimuendajú.

\textbf{Coleção Mundo Indígena} reúne materiais produzidos com pensadores de diferentes povos indígenas e pessoas que pesquisam, trabalham ou lutam pela garantia de seus direitos. Os livros foram feitos para serem utilizados pelas comunidades envolvidas na sua produção, e por isso uma parte significativa das obras é bilíngue. Esperamos divulgar a imensa diversidade linguística dos povos indígenas no Brasil, que compreende mais de 150 línguas pertencentes a mais de trinta famílias linguísticas.



